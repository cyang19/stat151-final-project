% Options for packages loaded elsewhere
\PassOptionsToPackage{unicode}{hyperref}
\PassOptionsToPackage{hyphens}{url}
%
\documentclass[
  12pt,
]{article}
\usepackage{amsmath,amssymb}
\usepackage{iftex}
\ifPDFTeX
  \usepackage[T1]{fontenc}
  \usepackage[utf8]{inputenc}
  \usepackage{textcomp} % provide euro and other symbols
\else % if luatex or xetex
  \usepackage{unicode-math} % this also loads fontspec
  \defaultfontfeatures{Scale=MatchLowercase}
  \defaultfontfeatures[\rmfamily]{Ligatures=TeX,Scale=1}
\fi
\usepackage{lmodern}
\ifPDFTeX\else
  % xetex/luatex font selection
  \setmainfont[]{Times New Roman}
\fi
% Use upquote if available, for straight quotes in verbatim environments
\IfFileExists{upquote.sty}{\usepackage{upquote}}{}
\IfFileExists{microtype.sty}{% use microtype if available
  \usepackage[]{microtype}
  \UseMicrotypeSet[protrusion]{basicmath} % disable protrusion for tt fonts
}{}
\makeatletter
\@ifundefined{KOMAClassName}{% if non-KOMA class
  \IfFileExists{parskip.sty}{%
    \usepackage{parskip}
  }{% else
    \setlength{\parindent}{0pt}
    \setlength{\parskip}{6pt plus 2pt minus 1pt}}
}{% if KOMA class
  \KOMAoptions{parskip=half}}
\makeatother
\usepackage{xcolor}
\usepackage[margin=1in]{geometry}
\usepackage{graphicx}
\makeatletter
\def\maxwidth{\ifdim\Gin@nat@width>\linewidth\linewidth\else\Gin@nat@width\fi}
\def\maxheight{\ifdim\Gin@nat@height>\textheight\textheight\else\Gin@nat@height\fi}
\makeatother
% Scale images if necessary, so that they will not overflow the page
% margins by default, and it is still possible to overwrite the defaults
% using explicit options in \includegraphics[width, height, ...]{}
\setkeys{Gin}{width=\maxwidth,height=\maxheight,keepaspectratio}
% Set default figure placement to htbp
\makeatletter
\def\fps@figure{htbp}
\makeatother
\setlength{\emergencystretch}{3em} % prevent overfull lines
\providecommand{\tightlist}{%
  \setlength{\itemsep}{0pt}\setlength{\parskip}{0pt}}
\setcounter{secnumdepth}{-\maxdimen} % remove section numbering
% ============================
% Global formatting
% ============================
\usepackage{setspace}
\doublespacing  % Body double-spaced

\usepackage[margin=1in]{geometry}
\usepackage{mathptmx}             % Times-like font

\setlength{\parindent}{0.5in}
\setlength{\parskip}{0pt}

% ============================
% Single-spaced section headings
% ============================
\usepackage{titlesec}

\titlespacing*{\section}{0pt}{12pt}{6pt}
\titlespacing*{\subsection}{0pt}{12pt}{6pt}

\titleformat{\section}
  {\singlespacing\normalfont\large\bfseries}{\thesection}{1em}{}

\titleformat{\subsection}
  {\singlespacing\normalfont\normalsize\bfseries}{\thesubsection}{1em}{}

% ============================
% Endnotes instead of footnotes
% ============================
\usepackage{endnotes}
\let\footnote=\endnote
\ifLuaTeX
  \usepackage{selnolig}  % disable illegal ligatures
\fi
\IfFileExists{bookmark.sty}{\usepackage{bookmark}}{\usepackage{hyperref}}
\IfFileExists{xurl.sty}{\usepackage{xurl}}{} % add URL line breaks if available
\urlstyle{same}
\hypersetup{
  hidelinks,
  pdfcreator={LaTeX via pandoc}}

\author{}
\date{\vspace{-2.5em}}

\begin{document}

\hypertarget{intro}{%
\section{Intro}\label{intro}}

In response to the COVID-19 Pandemic, the Chinese government instated a
``Zero-COVID Policy,'' one that emphasized heavy regulation of personal
freedoms with the goal of limiting COVID cases to zero. Much has been
said about China's policy-making during this time, including expository
journalism\footnote{Murong Xuecun. (2023, April 18). China's `zero
  Covid' policy was a mass imprisonment campaign. The Guardian.
  \url{https://www.theguardian.com/commentisfree/2023/apr/18/china-zero-covid-policy-xi-jinping}},
research\footnote{Yan, K., Jiang, S., Xia, L., Jin, T., Dai, A., Gu, C.,
  \& Li, A. (2025). China's zero-COVID policy and psychological
  distress: a spatial quasi-experimental design. Journal of Social
  Policy, 54(3), 1029--1046. \url{doi:10.1017/S0047279423000430}}, and
public protest.\footnote{Schifrin, N., \& Cebrián Aranda, T. (2022,
  November 28). Thousands in China protest zero-COVID policy in largest
  demonstrations in decades. PBS NewsHour.
  \url{https://www.pbs.org/newshour/show/thousands-in-china-protest-zero-covid-policy-in-largest-demonstrations-in-decades}}
Existing literature on China's repsonse to COVID-19 has largely
emphasized typical epidemiological effects including COVID-19
prevalence, respiratory outcomes, and more. We identified a gap of
longitudinal studies that seek to interrogate how COVID-19 impacted the
mental health of Chinese citizens during this time.

\noindent As such, we are primarily interested in the following two
research questions:

\begin{itemize}
\tightlist
\item
  What was the impact of COVID on depression in China?
\item
  Was that most caused by stringency, public health, or economic
  downturn?
\end{itemize}

\hypertarget{data}{%
\section{Data}\label{data}}

We combine data from a number of sources. Our foundational data is taken
from the
\href{https://cfpsdata.pku.edu.cn/#/home}{China Family Panel Survey (CFPS)}
out of Peking University. We use its survey waves in 2016, 2018, 2020,
and 2022; we only use those individuals who were surveyed at each of
these timepoints and were not missing crucial covariates or outcomes
(\(n=12806\)). Our outcome is given by the CES-D 20 survey, a
questionnaire that asks questions about a respondents mental health and
produces an output on a 60 point scale. The CFPS collected data on this
during each of these waves.

We merge this data with three different continuous treatment variables.
First, is the Oxford COVID-19 Government Response Tracker (OxCGRT) which
provides data on the stringency of government regulations during COVID.
It's data was collected by {[}BLANK{]}. Second, are public health
indices from the Open COVID Data Portal---specifically, we use BLANK
METRIC. Lastly, we use BLANK FOR ECONOMICS.

Our data has five levels: 4 time points nested in 12806 individuals in
7770 families in 3318 communities in 32 provinces.

\hypertarget{methods}{%
\section{Methods}\label{methods}}

Our data have time (month), \(t\), nested in an individual \(i\) who
belongs to a family \(f\) in a community \(c\) that is in one of China's
provinces \(p\).

Our stringency metric, \(\text{Treat}_{pt}\), ranges from 0 to 24 (TO BE
UPDATED) and is 0 for all months in surveys taken before 2020. For the
2020 and 2022 surveys, the stringency metric varies across provinces and
across months. Our multilevel model is: \begin{align*}
\text{Level 1 (time within person):    }
y_{tifcp} &= \pi_{ifcp} + \beta_{1}\,\text{Treat}_{pt} + \boldsymbol{\beta}_{Z}^{\top}\mathbf{Z}_{tifcp} + \varepsilon_{tifcp}, \\
\varepsilon_{tifcp} &\sim \mathcal{N}(0,\sigma^2_\varepsilon)\\
\text{Level 2 (person):    }
\pi_{ifcp} &= \beta_{fcp} + \boldsymbol{\beta}_{X}^{\top}\mathbf{X}_{ifcp} + u_{ifcp}, \\
u_{ifcp} &\sim \mathcal{N}(0,\sigma^2_i)\\
\text{Level 3 (family):    }
\beta_{fcp} &= \gamma_{cp} + u_{fcp}, \\
u_{fcp} &\sim \mathcal{N}(0,\sigma^2_f)\\
\text{Level 4 (community):     }
\gamma_{cp} &= \alpha_{p} + u_{cp}, \\
u_{cp} &\sim \mathcal{N}(0,\sigma^2_c),\\
\text{Level 5 (province):    }
\alpha_{p} &= \nu + u_{p}, \\
u_{p} &\sim \mathcal{N}(0,\sigma^2_p),
\end{align*} where \(y_{tifcp}\) is the depression score for individual
\(i\) in family \(f\), community \(c\), province \(p\) at month \(t\),
\(\text{Treat}_{pt}\) is the province-by-month COVID policy stringency
index, and \(\mathbf{Z}_{tifcp}\) is a vector of time-varying,
individual-level covariates (e.g.,LIST SOME HERE age, income, whatever;
see Appendix Table A1 DO THE APPENDIX). The coefficient \(\beta_{1}\)
captures the average association between policy stringency and
depression, conditional on other covariates.

where \(\mathbf{X}_{ifcp}\) is a vector of time-invariant
individual-level covariates (e.g., LIST SOME HERE see Appendix Table A1)
and \(\boldsymbol{\beta}_{X}\) is the corresponding coefficient vector.
The random intercept \(u_{ifcp}\) captures unobserved, person-specific
heterogeneity. We also include \(u_{fcp}\) as a family-level random
intercept, \(u_{cp}\) as a community-level random intercept, and
\(u_{p}\) as a province-level random intercept. \(\nu\) is the grand
mean depression level.

Taken together, this specification allows us to estimate the
province-month stringency effect \(\beta_{1}\) on individual depression
while accounting for time-varying individual covariates, time-invariant
person-level covariates, and clustering of observations within persons,
families, communities, and provinces.

\hypertarget{results}{%
\section{Results}\label{results}}

\hypertarget{discussion}{%
\section{Discussion}\label{discussion}}

\hypertarget{conclusion}{%
\section{Conclusion}\label{conclusion}}

\pagebreak

\hypertarget{endnotes}{%
\section{Endnotes}\label{endnotes}}

\begingroup

\setstretch{1} \% optional: single-space the endnotes \theendnotes
\endgroup

\end{document}
